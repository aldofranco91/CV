\documentclass{muratcan_cv}

\setname{Aldo Ramón}{Franco Comas}
\setaddress{Madrid/España}
\setmobile{+34 639 606 821}
\setmail{aldo911012@gmail.com}
\setposition{Work Student} %ignored for now
\setlinkedinaccount{https://www.linkedin.com/in/aldofranco91} %you can play with color of the template (red is also nice..)
\setgithubaccount{https://github.com/aldofranco91} %you can play with color of the template (red is also nice..)
\setthemecolor{red} %you can play with color of the template (red is also nice..)

\begin{document}
%Set variables
%You can add sections, texts, explanations just by copying the style below. Replace the dummy texts "\lipsum[1][x-x]\par" with actual texts.
%Create header
\headerview
\vspace{1ex}
%Sections
%
% Summary
%\addblocktext{Summary}{%
%\lipsum[1][1-12]\ %replace this part with actual text
%}
%

% Experience
\section{Experiencia}
    %
    \datedexperience{\href{https://www.ncs-spain.com/ncssp/}{Network Centric Software, S.A, Madrid, España.}}{03/2020 - Actualidad} 
    \explanation{Consultor Externo en AXA Seguros S.A.}
    \explanationdetail{
     \coloredbullet\  Adquisición de datos de fuentes abiertas y aplicación de técnicas estadísticas, técnicas geoespaciales y algoritmos de aprendizaje automático, para resolver problemas comerciales y técnicos con impacto en las prioridades de la empresa. \\
     \coloredbullet\  Trabajar en estrecha colaboración con los ingenieros de datos en el diseño e implementación de procesos y modelos en la producción. \\
     \coloredbullet\ Contribuir a ampliar la cultura de datos de la empresa. \\
     \coloredbullet\  Tecnologías utilizadas:  R, Python, Scala, Spark y Azure. \\
     }
    %
    \datedexperience{\href{https://www.axa.es/}{AXA Seguros Generales S.A, Madrid, España.}}{09/2018 - 12/2020 } 
    \explanation{Becario.} 
    \explanationdetail{
     \coloredbullet\  Adquisición de datos de fuentes abiertas y aplicación de técnicas estadísticas, técnicas geoespaciales y algoritmos de aprendizaje automático, para resolver problemas comerciales y técnicos con impacto en las prioridades de la empresa. \\
     \coloredbullet\  Trabajar en estrecha colaboración con los ingenieros de datos en el diseño e implementación de procesos y modelos en la producción. \\
     \coloredbullet\  Tecnologías utilizadas: R y Python. \\
     }
  %
 \datedexperience{\href{https://www.uc3m.es/ss/Satellite/UC3MInstitucional/es/PortadaMiniSiteA/1371229065435/Departamento_de_Estadistica}{Universidad Carlos III de Madrid, España.}}{09/2017 - 09/2018} 
 \explanation{Profesor ayudante.} 
 \explanationdetail{
 	\coloredbullet\ Departamento de Estadística. \\
 	\coloredbullet\ Clases prácticas de R y Matlab. \\
 }
    %
\datedexperience{\href{http://www.one.cu/}{Oficina Nacional de Estadística e Información, La Habana, Cuba.}}{09/2014 - 09/2016} 
\explanation{Estadístico.} 
\explanationdetail{
	\coloredbullet\ Desarrollo y creación de encuestas a nivel nacional. \\
	\coloredbullet\  Limpieza de datos, análisis y visualización de los resultados obtenidos de estas encuestas. \\
	\coloredbullet\  Tecnologías utilizadas: R, SPSS y Visual FoxPro. 
}
  

%Education
\section{Educación} 
    \datedexperience
    {\href{https://www.uc3m.es/master/mathematical-engineering}{Universidad Carlos III de Madrid, España.}}{2017-2019} 
    \explanation{Máster en Ingeniería Matemática.} 
    % \explanationdetail{\coloredbullet\ % 
    % \lipsum[1][3-4]\par %replace this part with actual text
    % }
    \datedexperience
    {\href{https://www.escueladeempresa.com/cursos-universitarios-6-meses/curso-universitario-de-especializacion-en-introduccion-a-la-ciberseguridad/?ap=pago}{Escuela de Empresas, Madrid, España.}}{2019} 
    \explanation{Curso de Ciberseguridad.} 
%
    \datedexperience{\href{https://www.uc3m.es/master/big-data-analytics}{Universidad Carlos III de Madrid, España.}}{2016-2018} 
    \explanation{Máster en Big Data.} 
    \datedexperience{\href{http://www.uh.cu/matematica}{Universidad de La Habana, Cuba.}}{2010-2014}
    \explanation{Licenciado en Matemáticas.} 

    
%
% Skills
\section{Habilidades}
    %
    \newcommand{\skillone}{\createskill{Lenguajes de programación}{\textbf{\emph{Con experiencia:}} \ \  Python \cpshalf R \ \ \textbf{\emph{Familiar:}} \ \  Scala \cpshalf Matlab \cpshalf Latex \cpshalf HTML }}
    %
    \newcommand{\skilltwo}{\createskill{Desarrollo de software}{GIT}}
    %
    \newcommand{\skillthree}{\createskill{Frameworks \ \& \ Librerías}{Jupyter \cpshalf Caret \cpshalf Numpy \cpshalf Pandas \cpshalf Scikit-learn \cpshalf Keras}}
    %
    \newcommand{\skillfour}{\createskill{Idiomas}{\textbf{\emph{Nativo:}} \ \  Spanish \ \ \textbf{\emph{Medio:}} \ \ English \ \ }}
    %
    \createskills{\skillone, \skilltwo, \skillthree, \skillfour}
%

\newpage

% Accomplishments
\section{Logros}
 \datedexperience
 	{Tercer lugar en el 	\href{https://www.u-tad.com/alumnos-del-grado-ingenieria-software-u-tad-ganan-datathon-microsoft-espana/}{U-Tad Data Science Student Challenge.}}{2017}
 %\explanation{Master's Thesis Mathematical Engineering.} 
  \explanationdetail{
  	\coloredbullet\ Primera datathon de Microsoft en España. \\
  	\coloredbullet\ Análisis de datos y búsqueda de patrones en una gran base de datos anónima proporcionada por el Banco Santander.   
  }
%

%Education
\section{Proyectos} 
\datedexperience
	{\href{https://github.com/aldofranco91/TFM_Ing_Mat/blob/master/!!Tesis/Aldo_TFM.pdf}{Procedimiento de vecino más cercano con matrices de distancia parcialmente observadas.}}{2019} 
	\explanation{Tésis de Master en Ingeniería en Matemáticas.}
	\explanationdetail{
		\coloredbullet\ Diseño de un procedimiento k-NN cuando, por razones de tiempo de respuesta, costo computacional (por ejemplo, en conjuntos de datos extremadamente grandes) o pruebas destructivas, no sea posible calcular las distancias entre las nuevas observaciones y todas las observaciones en la muestra de entrenamiento.
	} \\
%
\datedexperience
	{\href{https://github.com/aldofranco91/TFM_Big_Data/blob/master/!!Tesis/Aldo_TFM.pdf}{Previsión de los precios de la electricidad en España}}{2018} 
	\explanation{Tésis de Master en Big Data.}
	\explanationdetail{
		\coloredbullet\ Previsión de precios de electricidad a corto, medio y largo plazo. Fue desarrollada una aplicación Shiny, montada en AWS que permite la visualización y descarga de dichas predicciones y sus intervalos de predicción.  
	}

\section{Publicaciones}
\datedexperience
{\href{https://www.funcas.es/wp-content/uploads/2021/06/An\%C3\%A1lisis-econom\%C3\%A9trico-y-big-data.pdf}{Análisis econométrico y big data.}}{2021} \\
%
\datedexperience
{\href{https://1library.co/document/y4x1l85z-encuesta-indicadores-prevencion-infeccion-vih-sida.html}{Análisis de Indicadores sobre la infección del VIH/SIDA en Cuba.}}{2015} 

%Footnote
%\createfootnote
\end{document}
